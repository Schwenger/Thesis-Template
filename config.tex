
\title{Underwater Basket Weaving: An Ontology}
\setauthor{Maximilian Schwenger}
\setcity{Saarbrücken}
\setsubmissiondate{24}{April}{2019}

\ifthenelse{\boolean{dissertation}}{
  % First examiner is also chair.
  \setexaminerone{Prof.~Dr.~Clara Hair}
  \setexaminertwo{Prof.~Dr.~Ernest Xavier Amination}
  \setexaminerthree{Prof.~Dr.~Brigitte Oard}
  % Fourth examiner is from academic staff.
  \setexaminerfour{Dr.~Susanne Taff}
  \setdean{Prof.~Dr.~David Ean}
  \setadvisor{PD~Dr.~Alberta Dorothee Visor}
  \setdefensedate{February 30, 2020}
}{
  % Usually also the supervisor
  \setexaminerone{Prof.~Albus Percival Wulfric Brian Gandalf}
  % Second reviewer
  \setexaminertwo{Prof.~Minerva McGonagall}
  % Usually the PhD student advising you throughout your thesis. May be empty
  \setadvisor{Hermione Granger}
  % Usually the head of the group or a post-doc.
  \setsupervisor{Prof.~Albus Percival Wulfric Brian Gandalf}
  \setthesiskind{Master}
  \setuniversity{Saarland University}
  \setdepartment{Department of Computer Science}
}


% Decide whether or not you want the beloved own on the title page.
\setboolean{titlepagelogo}{true}
% Determines the placement of the owl.  Hard to explain, try out both.
\setboolean{logobottom}{true}
% Determines whether or not you are writing a dissertation or a bachelor/master thesis.
% Has an impact on the frontmatter.
\setboolean{dissertation}{false}
% Turn night mode on or off.
\setboolean{nightmode}{false}
% Switch XeLaTeX on or off. Generally, if the xelatex version is running, it simplifies a couple things.  Changes fonts.
% If either of the font sets is fine for you, use xelatex until you encounter a problem.  You'll find more support online for regular latex.
\setboolean{xelatex}{false}
% Allows you to include a statement that the print version and the digital versions are congruent.
% This is required for the print versions at Saarland University.
\setboolean{printversion}{true}
% Shows or hides frames. Useful for finding overfull boxes.
\setboolean{showframes}{true}

% This is the marker that is used in text before cross references.  Empty by default can e.g. be `\MVRightarrow\thinspace ' for an arrow in front of it, such as `-> Definition 5'.
\newcommand*\intextcrossrefmarker{}
%\renewcommand*\intextcrossrefmarker{\MVRightarrow\thinspace}

