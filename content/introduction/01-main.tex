\chapter{Introduction}\label{introduction}

In this chapter, we give an overview over the commands available to you via the preamble and how they are meant to be used.

\section{Environments}\label{introduction:environments}

\begin{theorem}
A theorem. Used for results that are usually the main result of a section or chapter.
\end{theorem}

\begin{proof}
The proof of the above theorem.	
\end{proof}


\begin{lemma}
A lemma. Used for auxiliary statements that are used to prove a theorem.
\end{lemma}

\begin{proof}
The proof of the above lemma	
\end{proof}


\begin{proposition}
A proposition. Statements proved by someone else and that are merely repeated for the sake of completeness.
\end{proposition}

\begin{corollary}
A corollary. A simple conclusion of existing statements, does not require a proof.
\end{corollary}

\begin{remark}
A remark. Used for simple observations that do not require a formal proof.	
\end{remark}

\begin{example}
An example.	
\end{example}

\begin{construction}
A construction of some object.
\end{construction}

\section{Labels and References}

There are two interesting ways to typeset references, as exemplified after the following example:
\begin{example}\label{introduction:labels}
	This example is labeled with \lstinline|\label{introduction:labels}|
\end{example}

Either use the ordinary cleveref-way with \lstinline|\Cref{label}|, which produces \Cref{introduction:labels}.
Or, use the cross-ref version thereof, \ie, \lstinline!\CCref!. 
It will produce the regular reference plus a reference to the page on which the referenced object occurred in the margin.
E.g., the command \lstinline!\CCref{introduction:labels}! produces the output ``\CCref{introduction:labels}''.

\CCref{introduction:labels} and also \CCref{introduction:labels}.

Note that all \lstinline|\cref{~}| commands are overwritten by capitalized abbreviations. 
If you want to add support for a new kind of referencable\dots things, make sure to include a line like \lstinline!\crefname{announcement}{Anou.}{Anous.}! somewhere before you first use it, best in your preamble.

\section{Definitions}\label{introduction:definitions}

If you define a term, you may want to use the macro \lstinline{\introterm}, which produces a little note in the margin alerting the reader to the place of the \introterm{introterm}.
You may supply an optional argument in order to change the \introterm[Optional Arg]{text} appearing in the margin. If the command is \introterm*[Stars]{starred}, it will not be declared as definition in the margin par.

\section{ToDos}\label{introduction:todos}
You can defne\todo{Fix typo} todos either outline or inline. 

Both kinds will appear in the list of todos after the table of contents. You can remove the table by removing the respective line in \lstinline{frontmatter/frontmatter-main.tex}.
\todo[inline]{Insert example for inline todo.}
Additionally, a \introterm[Global Todo]{global todos} will not be visible at the point of definition but appear in the list of todos. Use this to remind yourself of general problems like consistency checks or vague tasks to be refined and placed later. \globaltodo{Remove list of todos when thesis is ready to go.}

\section{Page Frames}\label{introduction:pageframes}

Page frames help you to easily find overfull boxes and guides you when layouting. You can turn them off by change the respective \lstinline{\includepackage} in \lstinline{packages.tex}

\section{Tikz}

See \Cref{weight} and \Cref{styles}.

\begin{figure} \centering
\begin{tikzpicture}
	\node[p?] (player) at (0,0) {$v_0$};
	\node[p0] (player0) at (4,0) {$v_1$};
	\node[p1] (player1) at (8,0) {$v_2$};
	
	\path
		(player) edge[weight = 4 at .3 anchor south,elided] (player0)
		(player0) edge[weight = 4 at .7 anchor north,elided=.2] (player1);
\end{tikzpicture}
\caption{Showcasing tikz-styles \texttt{p?}, \texttt{p0}, \texttt{p1}, as well as \texttt{weight} and \texttt{elided}.}\label{weight}
\end{figure}

\begin{figure} \centering
\begin{tikzpicture}
	\node[p0] (player) at (0,0) {$v_0$};
	\node[p0] (player0) at (4,0) {$v_1$};
	\node[p1] (player1) at (8,0) {$v_2$};
	\node[p1] (player2) at (8,2) {$v_3$};
	
	\path
		(player) edge[tick={$x_1$} at .2,tick={$x_2$} at .4,tick={$x_3$} at .7] (player0)
		(player0) edge[crossed out] (player1)
		(player1) edge[tickleft={$x_4$} at .4] (player2);
\end{tikzpicture}
\caption{Showcasing tikz-styles \texttt{tick}, \texttt{tickleft}, and \texttt{crossedout}.}\label{styles}
\end{figure}

