\RequirePackage{ifthenx}
\RequirePackage{etoolbox}
\RequirePackage{marginnote}
\RequirePackage{mparhack}
\RequirePackage[ruled,vlined,linesnumbered]{algorithm2e}
\RequirePackage[Lenny]{fncychap}
\RequirePackage[headsepline=.6pt,footsepline=false]{scrlayer-scrpage}
\RequirePackage{scrhack} 
\RequirePackage[inline]{enumitem} 

%%%%%%%%%%%%%%%%%%%%%%%%%%%%%%%%%%%%%%%%%%%%%%%%%%%%%
%%%%%%%%%%%%%%%%%%%%%%%%%%%%%%%%%%%%%%%%%%%%%%%%%%%%%
%%% Prepare new commands
\newcommand*{\setauthor}[1]{\newcommand*{\AUTHOR}{#1}}
\newcommand*{\setexaminerone}[1]{\newcommand*{\EXAMINERONE}{#1}}
\newcommand*{\setexaminertwo}[1]{\newcommand*{\EXAMINERTWO}{#1}}
\newcommand*{\setexaminerthree}[1]{\newcommand*{\EXAMINERTHREE}{#1}}
\newcommand*{\setexaminerfour}[1]{\newcommand*{\EXAMINERFOUR}{#1}}
\newcommand*{\setadvisor}[1]{\newcommand*{\ADVISOR}{#1}}
\newcommand*{\setsupervisor}[1]{\newcommand*{\SUPERVISOR}{#1}}
\newcommand*{\setdean}[1]{\newcommand*{\DEAN}{#1}}
\newcommand*{\setdefensedate}[1]{\newcommand*{\DEFENSEDATE}{#1}}
\newcommand*{\setthesiskind}[1]{\newcommand*{\THESISKIND}{#1}}
\newcommand*{\setuniversity}[1]{\newcommand*{\UNIVERSITY}{#1}}
\newcommand*{\setdepartment}[1]{\newcommand*{\DEPARTMENT}{#1}}
\newcommand*{\setcity}[1]{\newcommand*{\CITY}{#1}}
\newcommand*{\setsubmissiondate}[3]{
  \newcommand*{\SUBMISSIONDAY}{#1}
  \newcommand*{\SUBMISSIONMONTH}{#2}
  \newcommand*{\SUBMISSIONYEAR}{#3}
}

% The following commands allow us to use the usual title command for declaring the title.
% The reason is that several editors set the project's title to the argument of \title.
% We store the usual definition of \title in \constructtitle, use it later and renew \title itself.
\let\constructtitle\title{}
\renewcommand*{\title}[1]{\newcommand*{\TITLE}{#1}}

%%%%%%%%%%%%%%%%%%%%%%%%%%%%%%%%%%%%%%%%%%%%%%%%%%%%%
%%%%%%%%%%%%%%%%%%%%%%%%%%%%%%%%%%%%%%%%%%%%%%%%%%%%%
%%% Define configurable bool flags

\newboolean{titlepagelogo}
\newboolean{logobottom}
\newboolean{dissertation}
\newboolean{nightmode}
\newboolean{xelatex}
\newboolean{printversion}
\newboolean{showframes}

%%%%%%%%%%%%%%%%%%%%%%%%%%%%%%%%%%%%%%%%%%%%%%%%%%%%%
%%%%%%%%%%%%%%%%%%%%%%%%%%%%%%%%%%%%%%%%%%%%%%%%%%%%%
%%% Include bool configurations

%%%%%%% LEAVE THE FOLLOWING LINES AS IS %%%%%%%
\newcommand*{\setauthor}[1]{\newcommand*{\AUTHOR}{#1}}
\newcommand*{\setexaminerone}[1]{\newcommand*{\EXAMINERONE}{#1}}
\newcommand*{\setexaminertwo}[1]{\newcommand*{\EXAMINERTWO}{#1}}
\newcommand*{\setexaminerthree}[1]{\newcommand*{\EXAMINERTHREE}{#1}}
\newcommand*{\setexaminerfour}[1]{\newcommand*{\EXAMINERFOUR}{#1}}
\newcommand*{\setadvisor}[1]{\newcommand*{\ADVISOR}{#1}}
\newcommand*{\setsupervisor}[1]{\newcommand*{\SUPERVISOR}{#1}}
\newcommand*{\setdean}[1]{\newcommand*{\DEAN}{#1}}
\newcommand*{\setdefensedate}[1]{\newcommand*{\DEFENSEDATE}{#1}}
\newcommand*{\setthesiskind}[1]{\newcommand*{\THESISKIND}{#1}}
\newcommand*{\setuniversity}[1]{\newcommand*{\UNIVERSITY}{#1}}
\newcommand*{\setdepartment}[1]{\newcommand*{\DEPARTMENT}{#1}}
\newcommand*{\setcity}[1]{\newcommand*{\CITY}{#1}}
\newcommand*{\setsubmissiondate}[3]{
  \newcommand*{\SUBMISSIONDAY}{#1}
  \newcommand*{\SUBMISSIONMONTH}{#2}
  \newcommand*{\SUBMISSIONYEAR}{#3}
}

% The following commands allow us to use the usual title command for declaring the title.
% The reason is that several editors set the project's title to the argument of \title.
% We store the usual definition of \title in \constructtitle, use it later and renew \title itself.
\let\constructtitle\title{}
\renewcommand*{\title}[1]{\newcommand*{\TITLE}{#1}}

\newboolean{titlepagelogo}
\newboolean{logobottom}
\newboolean{dissertation}
\newboolean{nightmode}
\newboolean{xelatex}
\newboolean{printversion}
\newboolean{showframes}


%%%%%%% START CONFIGURING HERE %%%%%%%

% Decide whether or not you want the beloved own on the title page.
\setboolean{titlepagelogo}{true}
% Determines the placement of the owl.  Hard to explain, try out both.
\setboolean{logobottom}{true}
% Determines whether or not you are writing a dissertation or a bachelor/master thesis.
% Has an impact on the frontmatter.
\setboolean{dissertation}{false}
% Turn night mode on or off.
\setboolean{nightmode}{false}
% Switch XeLaTeX on or off. Generally, if the xelatex version is running, it simplifies a couple things.  Changes fonts.
% If either of the font sets is fine for you, use xelatex until you encounter a problem.  You'll find more support online for regular latex.
\setboolean{xelatex}{false}
% Allows you to include a statement that the print version and the digital versions are congruent.
% This is required for the print versions at Saarland University.
\setboolean{printversion}{true}
% Shows or hides frames. Useful for finding overfull boxes.
\setboolean{showframes}{true}

% This is the marker that is used in text before cross references.  Empty by default can e.g. be `\MVRightarrow\thinspace ' for an arrow in front of it, such as `-> Definition 5'.
\newcommand*\intextcrossrefmarker{}
% \renewcommand*\intextcrossrefmarker{\MVRightarrow\thinspace}

\title{Underwater Basket Weaving: An Ontology}
\setauthor{Maximilian Schwenger}
\setcity{Saarbrücken}
\setsubmissiondate{24}{April}{2019}

\ifthenelse{\boolean{dissertation}}{
  % First examiner is also chair.
  \setexaminerone{Prof.~Dr.~Clara Hair}
  \setexaminertwo{Prof.~Dr.~Ernest Xavier Amination}
  \setexaminerthree{Prof.~Dr.~Brigitte Oard}
  % Fourth examiner is from academic staff.
  \setexaminerfour{Dr.~Susanne Taff}
  \setdean{Prof.~Dr.~David Ean}
  \setadvisor{PD~Dr.~Alberta Dorothee Visor}
  \setdefensedate{February 30, 2020}
}{
  % Usually also the supervisor
  \setexaminerone{Prof.~Albus Percival Wulfric Brian Gandalf}
  % Second reviewer
  \setexaminertwo{Prof.~Minerva McGonagall}
  % Usually the PhD student advising you throughout your thesis. May be empty
  \setadvisor{Hermione Granger}
  % Usually the head of the group or a post-doc.
  \setsupervisor{Prof.~Albus Percival Wulfric Brian Gandalf}
  \setthesiskind{Master}
  \setuniversity{Saarland University}
  \setdepartment{Department of Computer Science}
}



%%%%%%%%%%%%%%%%%%%%%%%%%%%%%%%%%%%%%%%%%%%%%%%%%%%%%
%%%%%%%%%%%%%%%%%%%%%%%%%%%%%%%%%%%%%%%%%%%%%%%%%%%%%
%%% Import geometry 

% The first shows the dimensions, makes it easier to detect inconsistencies. Switch to latter before submission.
\ifthenelse{\boolean{showframes}}{
	\usepackage[showframe]{geometry}
}{\usepackage{geometry}}


%%%%%%%%%%%%%%%%%%%%%%%%%%%%%%%%%%%%%%%%%%%%%%%%%%%%%
%%%%%%%%%%%%%%%%%%%%%%%%%%%%%%%%%%%%%%%%%%%%%%%%%%%%%
%%% Patches and Hacks 

% Fix misaligned line numbers:
% https://tex.stackexchange.com/questions/476579/how-to-align-line-numbers-horizontally-with-package-algorithm2e
\makeatletter
\patchcmd\algocf@Vline{\vrule}{\vrule \kern-0.4pt}{}{}
\patchcmd\algocf@Vsline{\vrule}{\vrule \kern-0.4pt}{}{}
\makeatother

% Fix for deprecated font command
\DeclareOldFontCommand{\sf}{\normalfont\sffamily}{\mathsf}
\DeclareOldFontCommand{\rm}{\normalfont\rmfamily}{\mathsf}

% Ugly hack to get useful \vdots, taken from http://tex.stackexchange.com/a/112212
\makeatletter
\DeclareRobustCommand{\rvdots}{%
  \vbox{
    \baselineskip4\p@\lineskiplimit\z@
    \kern-\p@
    \hbox{.}\hbox{.}\hbox{.}
  }}
\makeatother


%%%%%%%%%%%%%%%%%%%%%%%%%%%%%%%%%%%%%%%%%%%%%%%%%%%%%
%%%%%%%%%%%%%%%%%%%%%%%%%%%%%%%%%%%%%%%%%%%%%%%%%%%%%
%%% Labels and References

\newcommand{\MarginCrossRef}[1]{%
\marginnote{%
  \scriptsize \MVRightarrow\,%
  \text{\cref{#1}, p.~\pageref{#1}}}%
}

\renewcommand*{\raggedleftmarginnote}{} % Fix spacing in margin notes.

\newcommand{\CCref}[1]{\intextcrossrefmarker{}\Cref{#1}\MarginCrossRef{#1}}

\NewDocumentCommand{\introterm}{ s o m }{%
  \marginnote{\scriptsize\flushleft%
    \IfBooleanTF{#1}{}{\textbf{Def.}} % Add Def. only if star is not there.
    \IfValueTF{#2}{#2}{#3}% Take optional 2nd arg if it exists, otherwise 3rd.
  }%
  \emph{#3}%
}


%%%%%%%%%%%%%%%%%%%%%%%%%%%%%%%%%%%%%%%%%%%%%%%%%%%%%
%%%%%%%%%%%%%%%%%%%%%%%%%%%%%%%%%%%%%%%%%%%%%%%%%%%%%
%%% Theorem Environments

\newcommand*\Vhrulefill[1]{\leavevmode\leaders\hrule height 1ex depth \dimexpr#1-1ex\hfill}
\newcommand*{\defheaderline}{{\color{lightgray}\Vhrulefill{1pt}}\par}


%%%%%%%%%%%%%%%%%%%%%%%%%%%%%%%%%%%%%%%%%%%%%%%%%%%%%
%%%%%%%%%%%%%%%%%%%%%%%%%%%%%%%%%%%%%%%%%%%%%%%%%%%%%
%%% Templates for the Statement and Congruence pages

\newcommand{\STATEMENTPAGE}{
  \vspace*{2cm}

  \noindent\textbf{Eidesstattliche Erkl\"arung}

  \noindent Ich erkl\"are hiermit an Eides Statt, dass ich die vorliegende Arbeit
  selbst\"andig verfasst und keine anderen als die angegebenen Quellen und
  Hilfsmittel verwendet habe.

  \bigskip
  \noindent\textbf{Statement in Lieu of an Oath}

  \noindent I hereby confirm that I have written this thesis on my own and that I
  have not used any other media or materials than the ones referred to in
  this thesis.


  \vspace{5cm}

  \noindent\textbf{Einverst\"andniserkl\"arung}

  \noindent Ich bin damit einverstanden, dass meine (bestandene) Arbeit in beiden
  Versionen in die Bibliothek der Informatik aufgenommen und damit
  ver\"offentlicht wird.

  \bigskip
  \noindent\textbf{Declaration of Consent}

  \noindent I agree to make both versions of my thesis (with a passing grade)
  accessible to the public by having them added to the library of the
  Computer Science Department.

  \vspace*{5cm}
  \noindent\rule{8cm}{.4pt}\\[\medskipamount]
  \CITY, \SUBMISSIONDAY\ \SUBMISSIONMONTH, \SUBMISSIONYEAR{}

  \cleardoublepage{}
}

\newcommand{\CONGRUENCESTATEMENT}{
  \vspace*{2cm}

  \noindent\textbf{Erkl\"arung}

  \noindent Ich erkläre hiermit, dass die vorliegende Arbeit mit der elektronischen Version übereinstimmt.

  \bigskip
  
  \noindent\textbf{Statement}

  \noindent I hereby confirm the congruence of the contents of the printed data and the electronic version of the thesis.

  \vspace*{3cm}
  \noindent\rule{8cm}{.4pt}\\[\medskipamount]
  \CITY, \SUBMISSIONDAY\ \SUBMISSIONMONTH, \SUBMISSIONYEAR{}

  \cleardoublepage{}
}

%%%%%%%%%%%%%%%%%%%%%%%%%%%%%%%%%%%%%%%%%%%%%%%%%%%%%
%%%%%%%%%%%%%%%%%%%%%%%%%%%%%%%%%%%%%%%%%%%%%%%%%%%%%
%%% Adjust some values and lengths

% Punish widows and orphans because we're awful people:
\widowpenalty10000
\clubpenalty10000

% Enable emergency stretches because we don't mind a few empty points.
\emergencystretch=10pt

% Set the margin par width appropriately
\setlength{\marginparwidth}{2.8cm} 

% Increase distance between array rows
\renewcommand\arraystretch{1.5}
\setlength{\jot}{8pt}

\setdescription{topsep=10pt,itemsep=0pt}

%%%%%%%%%%%%%%%%%%%%%%%%%%%%%%%%%%%%%%%%%%%%%%%%%%%%%
%%%%%%%%%%%%%%%%%%%%%%%%%%%%%%%%%%%%%%%%%%%%%%%%%%%%%
%%% Font Style

\ifthenelse{\boolean{xelatex}}{
  \usepackage{fontspec}
  % \usepackage{zi4} % Nice monospace font
  \usepackage[scaled=0.84]{beramono}
  \setmainfont[ 
      BoldFont       = texgyrepagella-bold.otf,
      ItalicFont     = texgyrepagella-italic.otf,
      BoldItalicFont = texgyrepagella-bolditalic.otf ]
    {texgyrepagella-regular.otf}
  \usepackage{unicode-math}
  \setmathfont{Asana-Math.otf}
}{
  \usepackage[utf8]{inputenc}
  \usepackage[T1]{fontenc}
  \usepackage[scaled=0.88]{beraserif}
  \usepackage[scaled=0.85]{berasans}
  \usepackage[scaled=0.84]{beramono}
  \usepackage{newpxtext}
  \usepackage[T1,small,euler-digits]{eulervm} % Euler Math Font
  \linespread{1.05}
}



%%%%%%%%%%%%%%%%%%%%%%%%%%%%%%%%%%%%%%%%%%%%%%%%%%%%%
%%%%%%%%%%%%%%%%%%%%%%%%%%%%%%%%%%%%%%%%%%%%%%%%%%%%%
%%% Headers and Footers

% Change \leftmark for chapters to be of form e.g. `1. Introduction`.
\renewcommand{\chaptermark}[1]{\markboth{\thechapter.~#1}{}}
% Change \rightmark for sections to be of form e.g. `1.2. Related Work`.
\renewcommand{\sectionmark}[1]{\markright{\thesection.~#1}}

% Set chapter numbers to Helvetica.
\ChNumVar{\fontsize{60}{62}\usefont{OT1}{phv}{m}{n}\selectfont}
% \ChNumVar{\fontsize{60}{62}\normalfont}

\ifthenelse{\boolean{xelatex}}{
  \ChTitleVar{\fontsize{30}{33}\bfseries\sffamily}
}{
  \ChTitleVar{\fontsize{30}{33}\bfseries\sffamily\selectfont}
}

\renewcommand{\chaptermark}[1]{%
 \markleft{\normalfont\sffamily\thechapter.\ \scshape#1}{}}
\renewcommand{\sectionmark}[1]{%
 \markright{\normalfont\sffamily\thesection.\ #1}{}}
%  \renewcommand{\sectionmark}[1]{%
%  \markboth{\thesection.\ #1 Whatwhat}{}}
% On even pages, put the chapter title into the left header
\lehead{\leftmark}
% On odd pages, put the section title into the right header
\rohead{\normalfont\sffamily\rightmark}
% Clear footer right and left, put page number in center
\lefoot*{}
\rofoot*{}
\cfoot*{\pagemark}

%%%%%%%%%%%%%%%%%%%%%%%%%%%%%%%%%%%%%%%%%%%%%%%%%%%%%
%%%%%%%%%%%%%%%%%%%%%%%%%%%%%%%%%%%%%%%%%%%%%%%%%%%%%
%%% Style title page

\date{\CITY, \SUBMISSIONMONTH\ \SUBMISSIONYEAR}
\constructtitle{\vspace{\pretitlewidth}\Huge\TITLE\vspace{\titlesepwidth}}
\ifthenelse{\boolean{dissertation}}{
  \author{Dissertation zur Erlangung des Grades \\ 
  des Doktors der Naturwissenschaften \\
  der Fakultät für Mathematik und Informatik \\
  der Universität des Saarlandes \\[\authorsepwidth]
  vorgelegt von  \\
  \AUTHOR}
}{
  \author{%
    \Large %
    \UNIVERSITY\\[\medskipamount]
    \DEPARTMENT\\[.9\authorsepwidth]
    \textsc{\THESISKIND's Thesis} \\[.9\authorsepwidth]
    \emph{submitted by} \\
    \AUTHOR\\[.9\authorsepwidth]
  }
}

%%%%%%%%%%%%%%%%%%%%%%%%%%%%%%%%%%%%%%%%%%%%%%%%%%%%%
%%%%%%%%%%%%%%%%%%%%%%%%%%%%%%%%%%%%%%%%%%%%%%%%%%%%%
%%% Style the Logo on the Front Page

\ifthenelse{\boolean{logobottom}}{
  \newcommand*\logox{0}
  \newcommand*\logoy{-350}
  \newcommand*\logoopacity{.3}
  \newcommand*\authorsepwidth{2cm}
  \newcommand*\titlesepwidth{1cm}
  \newcommand*\pretitlewidth{0cm}
}{
  \newcommand*\logox{0}
  \newcommand*\logoy{0}
  \newcommand*\logoopacity{.1}
  \newcommand*\authorsepwidth{1cm}
  \newcommand*\titlesepwidth{0cm}
  \newcommand*\pretitlewidth{10cm}
}

%%%%%%%%%%%%%%%%%%%%%%%%%%%%%%%%%%%%%%%%%%%%%%%%%%%%%
%%%%%%%%%%%%%%%%%%%%%%%%%%%%%%%%%%%%%%%%%%%%%%%%%%%%%
%%% Let figures and tables share counter

\makeatletter
\let\c@table\c@figure{}
\makeatother 


%%%%%%%%%%%%%%%%%%%%%%%%%%%%%%%%%%%%%%%%%%%%%%%%%%%%%
%%%%%%%%%%%%%%%%%%%%%%%%%%%%%%%%%%%%%%%%%%%%%%%%%%%%%
%%% Style ToC and Header

% Set plain page style for ToC/LoF/LoT spanning more than one page.
\AfterTOCHead{\pagestyle{plain}}
\AfterStartingTOC{\clearpage}

% Style ToC/... Header:
% Store the name in old macro to be used in styled version.
\let\oldcontentsname\contentsname{}
\renewcommand{\contentsname}{%
  \huge\sf \rule{\textwidth}{.8pt}\\[\medskipamount]%
  \oldcontentsname\\[-2\medskipamount]
  \huge\sf \rule{\textwidth}{.8pt}
}
\let\oldlistfigurename\listfigurename{}
\renewcommand{\listfigurename}{%
  \huge\sf \rule{\textwidth}{.8pt}\\[\medskipamount]%
  \oldlistfigurename\\[-2\medskipamount]
  \huge\sf \rule{\textwidth}{.8pt}
}
\let\oldlisttablename\listtablename{}
\renewcommand{\listtablename}{%
  \huge\sf \rule{\textwidth}{.8pt}\\[\medskipamount]%
  \oldlisttablename\\[-2\medskipamount]
  \huge\sf \rule{\textwidth}{.8pt}
}


%%%%%%%%%%%%%%%%%%%%%%%%%%%%%%%%%%%%%%%%%%%%%%%%%%%%%
%%%%%%%%%%%%%%%%%%%%%%%%%%%%%%%%%%%%%%%%%%%%%%%%%%%%%
%%% Background image for title page

% Produce background picture on title page featuring the logo of the university.
\makeatletter
\newcommand{\@printlogo}[1]{
\AddToShipoutPicture*{
    \put(\logox,\logoy){
      \parbox[b][\paperheight]{\paperwidth}{%
        \vfill
        \centering
        \begin{tikzpicture}
          \node [opacity=\logoopacity] (0,0) {
            \includegraphics[%
              height=.9\paperheight,%
              width=.9\paperwidth,%
              keepaspectratio,%
            ]{#1}
          };
        \end{tikzpicture}
        \vfill
      }
    }
  }
}
\newcommand{\printlogotransparentbg}{\@printlogo{images/logo.png}}
\newcommand{\printlogoopaquebg}{\@printlogo{images/logo.pdf}}
\makeatother


