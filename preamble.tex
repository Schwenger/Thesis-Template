%%%%%%%%%%%%%%%%%%%%%%%%%%%%%%%%%%%%%%%%%%%%%%%%%%%%%
%%%%%%%%%%%%%%%%%%%%%%%%%%%%%%%%%%%%%%%%%%%%%%%%%%%%%
%%% Fix for deprecated font command
\DeclareOldFontCommand{\sf}{\normalfont\sffamily}{\mathsf}
\DeclareOldFontCommand{\rm}{\normalfont\rmfamily}{\mathsf}

%%%%%%%%%%%%%%%%%%%%%%%%%%%%%%%%%%%%%%%%%%%%%%%%%%%%%
%%%%%%%%%%%%%%%%%%%%%%%%%%%%%%%%%%%%%%%%%%%%%%%%%%%%%
%%% Headers and Footers

% Change \leftmark for chapters to be of form e.g. `Chapter 1. Introduction`.
\renewcommand{\chaptermark}[1]{\markboth{\chaptername~\thechapter.~ #1}{}}
% Change \rightmark for sections to be of form e.g. `1.2. Related Work`.
\renewcommand{\sectionmark}[1]{\markright{\thesection. ~#1}}

% Header is now in small caps and larger.
\addtokomafont{pagehead}{\large\scshape}
% Increase space between sepline and footer.
\addtokomafont{pagefoot}{\scshape\raisebox{-8pt}}

% On even pages, put the chapter title into the left header
\lehead{\leftmark}
% On odd pages, put the section title into the right header
\rohead{\rightmark}
% Put the page number on the left side on even pages and on the right side on odd pages
\lefoot{\thepage}
\rofoot{\thepage}

%%%%%%%%%%%%%%%%%%%%%%%%%%%%%%%%%%%%%%%%%%%%%%%%%%%%%
%%%%%%%%%%%%%%%%%%%%%%%%%%%%%%%%%%%%%%%%%%%%%%%%%%%%%
%%% Labels and References

% CrossRef and SecCrossRef are not to be used in the main document, but are only used for the concise definition of the following commands
\newcommand{\CrossRef}[1]{%
\marginpar{%
	\footnotesize \MVRightarrow\,%
	\text{Sec.~\sectionref{#1}, Page~\pageref{#1}}}%
}
\newcommand{\SecCrossRef}[1]{%
\marginpar{%
	\footnotesize \MVRightarrow\,%
	\text{Page~\pageref{#1}}}%
}


\newcommand{\seclabel}[1]{\label{sec:#1}}
\newcommand{\secref}[1]{Section~\ref{sec:#1}}
\newcommand{\Csecref}[1]{\MVRightarrow\thinspace Section~\ref{sec:#1}\SecCrossRef{sec:#1}}
\newcommand{\chapref}[1]{Chapter~\ref{sec:#1}}
\newcommand{\Cchapref}[1]{\MVRightarrow\thinspace Chapter~\ref{sec:#1}\SecCrossRef{sec:#1}}

\newcommand{\proplabel}[1]{\label{prop:#1}}
\newcommand{\propref}[1]{Proposition~\ref{prop:#1}}
\newcommand{\subpropref}[2]{Proposition~\ref{prop:#1}.\ref{prop:#1:#2}}
\newcommand{\Cpropref}[1]{\MVRightarrow\thinspace  Proposition~\ref{prop:#1}\CrossRef{prop:#1}}
\newcommand{\Csubpropref}[2]{\MVRightarrow\thinspace  Proposition~\ref{prop:#1}.\ref{prop:#1:#2}\CrossRef{prop:#1:#2}}

\newcommand{\lemlabel}[1]{\label{lem:#1}}
\newcommand{\lemref}[1]{Lemma~\ref{lem:#1}}
\newcommand{\sublemref}[2]{Lemma~\ref{lem:#1}.\ref{lem:#1:#2}}
\newcommand{\Clemref}[1]{\MVRightarrow\thinspace Lemma~\ref{lem:#1}\CrossRef{lem:#1}}
\newcommand{\Csublemref}[2]{\MVRightarrow\thinspace Lemma~\ref{lem:#1}.\ref{lem:#1:#2}\CrossRef{lem:#1:#2}}

\newcommand{\corlabel}[1]{\label{cor:#1}}
\newcommand{\corref}[1]{Corollary~\ref{cor:#1}}
\newcommand{\subcorref}[2]{Corollary~\ref{cor:#1}.\ref{cor:#1:#2}}
\newcommand{\Ccorref}[1]{\MVRightarrow\thinspace Corollary~\ref{cor:#1}\CrossRef{cor:#1}}
\newcommand{\Csubcorref}[2]{\MVRightarrow\thinspace Corollary~\ref{cor:#1}.\ref{cor:#1:#2}\CrossRef{cor:#1}}

\newcommand{\thmlabel}[1]{\label{thm:#1}}
\newcommand{\thmref}[1]{Theorem~\ref{thm:#1}}
\newcommand{\subthmref}[2]{Theorem~\ref{thm:#1}.\ref{thm:#1:#2}}
\newcommand{\Cthmref}[1]{\MVRightarrow\thinspace Theorem~\ref{thm:#1}\CrossRef{thm:#1}}
\newcommand{\Csubthmref}[2]{\MVRightarrow\thinspace Theorem~\ref{thm:#1}.\ref{thm:#1:#2}\CrossRef{thm:#1:#2}}

\newcommand{\remlabel}[1]{\label{rem:#1}}
\newcommand{\remref}[1]{Remark~\ref{rem:#1}}
\newcommand{\subremref}[2]{Remark~\ref{rem:#1}.\ref{rem:#1:#2}}
\newcommand{\Cremref}[1]{\MVRightarrow Remark~\ref{rem:#1}\CrossRef{rem:#1}}
\newcommand{\Csubremref}[2]{\MVRightarrow\thinspace Remark~\ref{rem:#1}.\ref{rem:#1:#2}\CrossRef{rem:#1:#2}}

\newcommand{\exmlabel}[1]{\label{exm:#1}}
\newcommand{\exmref}[1]{Example~\ref{exm:#1}}
\newcommand{\Cexmref}[1]{\MVRightarrow\thinspace Example~\ref{exm:#1}\CrossRef{exm:#1}}

\newcommand{\figlabel}[1]{\label{fig:#1}}
\newcommand{\figref}[1]{Figure~\ref{fig:#1}}
\newcommand{\subfigref}[2]{Figure~\ref{fig:#1:#2}}
\newcommand{\Cfigref}[1]{\MVRightarrow\thinspace Figure~\ref{fig:#1}\CrossRef{fig:#1}}
\newcommand{\Csubfigref}[2]{\MVRightarrow\thinspace Figure~\ref{fig:#1:#2}\CrossRef{fig:#1:#2}}

\newcommand{\tablabel}[1]{\label{tab:#1}}
\newcommand{\tabref}[1]{Table~\ref{tab:#1}}
\newcommand{\Ctabref}[1]{\MVRightarrow\thinspace Table~\ref{tab:#1}\CrossRef{tab:#1}}

\newcommand{\alglabel}[1]{\label{alg:#1}}
\newcommand{\algref}[1]{Algorithm~\ref{alg:#1}}
\newcommand{\Calgref}[1]{\MVRightarrow\thinspace Algorithm~\ref{alg:#1}\CrossRef{alg:#1}}

%%%%%%%%%%%%%%%%%%%%%%%%%%%%%%%%%%%%%%%%%%%%%%%%%%%%%
%%%%%%%%%%%%%%%%%%%%%%%%%%%%%%%%%%%%%%%%%%%%%%%%%%%%%
%%% Theorem Environments

\newtheorem{theorem}{Theorem}[chapter]
\newtheorem{lemma}[theorem]{Lemma}
\newtheorem{proposition}[theorem]{Proposition}
\newtheorem{corollary}[theorem]{Corollary}
\newtheorem{remark}[theorem]{Remark}

\theoremstyle{definition}
% Taken from https://tex.stackexchange.com/a/32394
\newtheorem{examplex}[theorem]{Example}
\newenvironment{example}
  {\pushQED{\qed}\renewcommand{\qedsymbol}{$\triangle$}\examplex}
  {\popQED\endexamplex}
\newtheorem{construction}{Construction}[chapter]

%%%%%%%%%%%%%%%%%%%%%%%%%%%%%%%%%%%%%%%%%%%%%%%%%%%%%
%%%%%%%%%%%%%%%%%%%%%%%%%%%%%%%%%%%%%%%%%%%%%%%%%%%%%
%%% Let figures and tables share counter

\makeatletter
\let\c@table\c@figure
\makeatother 

%%%%%%%%%%%%%%%%%%%%%%%%%%%%%%%%%%%%%%%%%%%%%%%%%%%%%
%%%%%%%%%%%%%%%%%%%%%%%%%%%%%%%%%%%%%%%%%%%%%%%%%%%%%
%%% Customization of algorithm

\renewcommand{\algorithmicrequire}{\textbf{Input:}}
\renewcommand{\algorithmicensure}{\textbf{Output:}}
\renewcommand{\algorithmiccomment}[1]{\hfill{\textcolor{gray}{ /* #1 */ }}}

%%%%%%%%%%%%%%%%%%%%%%%%%%%%%%%%%%%%%%%%%%%%%%%%%%%%%
%%%%%%%%%%%%%%%%%%%%%%%%%%%%%%%%%%%%%%%%%%%%%%%%%%%%%
%%% Other Environments

\newcommand{\contribution}[1]{#1}
\newenvironment{decisionproblem}{\begin{quote}``\!\!}{\!\!''\end{quote}}
\newenvironment{invariant}{\begin{quotation}}{\end{quotation}}

%%%%%%%%%%%%%%%%%%%%%%%%%%%%%%%%%%%%%%%%%%%%%%%%%%%%%
%%%%%%%%%%%%%%%%%%%%%%%%%%%%%%%%%%%%%%%%%%%%%%%%%%%%%
%%% Ugly hack to get useful \vdots, taken from http://tex.stackexchange.com/a/112212

\makeatletter
\DeclareRobustCommand{\rvdots}{%
  \vbox{
    \baselineskip4\p@\lineskiplimit\z@
    \kern-\p@
    \hbox{.}\hbox{.}\hbox{.}
  }}
\makeatother

%%%%%%%%%%%%%%%%%%%%%%%%%%%%%%%%%%%%%%%%%%%%%%%%%%%%%
%%%%%%%%%%%%%%%%%%%%%%%%%%%%%%%%%%%%%%%%%%%%%%%%%%%%%
%%% Formatting Help

%\newcommand{\definition}[2][]{\marginpar{\footnotesize #1}\emph{#2}}
\newcommand{\definition}[2][]{%
	\marginpar{\footnotesize\flushleft\textbf{Def. }%
		\ifthenelse{\equal{#1}{}}%
			{#2} % If-branch
			{#1} % Else-branch
		}
	\textsc{#2}%
}


%%%%%%%%%%%%%%%%%%%%%%%%%%%%%%%%%%%%%%%%%%%%%%%%%%%%%
%%%%%%%%%%%%%%%%%%%%%%%%%%%%%%%%%%%%%%%%%%%%%%%%%%%%%
%%% Checkmarks

\newcommand{\cmark}{{\color{mydarkgreen}{\ding{51}}}}%
\newcommand{\ucmark}{{\color{gray}{\raisebox{-1pt}{\ding{55}}}}}%


%%%%%%%%%%%%%%%%%%%%%%%%%%%%%%%%%%%%%%%%%%%%%%%%%%%%%
%%%%%%%%%%%%%%%%%%%%%%%%%%%%%%%%%%%%%%%%%%%%%%%%%%%%%
%%% Acceptance/Winning Conditions

\newcommand{\safety}[1]{\ensuremath{\textsc{Safety}(#1)}}
\newcommand{\buchi}[1]{\ensuremath{\textsc{Büchi}(#1)}}
\newcommand{\parity}[1]{\ensuremath{\textsc{Parity}(#1)}}
\newcommand{\energyparity}[2]{\ensuremath{\textsc{EnergyParity}_{#1}(#2)}}
\newcommand{\bndparity}[1]{\ensuremath{\textsc{BndParity}(#1)}}
\newcommand{\finparity}[1]{\ensuremath{\textsc{FinParity}(#1)}}
\newcommand{\costparity}[1]{\ensuremath{\textsc{CostParity}(#1)}}
\newcommand{\bcostparity}[2]{\ensuremath{\textsc{CostParity}_{#1}(#2)}}
\newcommand{\weightparity}[1]{\ensuremath{\textsc{WeightParity}(#1)}}
\newcommand{\bweightparity}[2]{\ensuremath{\textsc{WeightParity}_{#1}(#2)}}
\newcommand{\bndweightparity}[1]{\ensuremath{\textsc{BndWeightParity}(#1)}}
\newcommand{\meanpayoffparity}[1]{\ensuremath{\textsc{MeanPayoffParity}(#1)}}

\newcommand{\countdown}[1]{\ensuremath{\textsc{Countdown}(#1)}}

%%%%%%%%%%%%%%%%%%%%%%%%%%%%%%%%%%%%%%%%%%%%%%%%%%%%%
%%%%%%%%%%%%%%%%%%%%%%%%%%%%%%%%%%%%%%%%%%%%%%%%%%%%%
%%% Complexity Classes

\newcommand{\logspace}{\textsc{LogSpace}\xspace}
\newcommand{\ptime}{\textsc{PTime}\xspace}
\newcommand{\up}{\textsc{UP}\xspace}
\newcommand{\coup}{\textsc{coUP}\xspace}
\newcommand{\np}{\textsc{NP}\xspace}
\newcommand{\conp}{\textsc{coNP}\xspace}
\newcommand{\aptime}{\textsc{APTime}\xspace}
\newcommand{\pspace}{\textsc{PSpace}\xspace}
\newcommand{\exptime}{\textsc{ExpTime}\xspace}
\newcommand{\expspace}{\textsc{ExpSpace}\xspace}